\chapter{Estudio de mercado}
\section{Objetivo}
Mediante el Estudio de Mercado se pretende atender un grupo de personas con necesidades por
satisfacer, los cuales cuentan con ingresos y voluntad para invertir (gastar).

Para lograr lo anterior, la satisfacción del cliente (consumidor) se debe de tener presente
las siguentes custiones:
\begin{itemize}
    \item Descubir necesidades actuales y futuras de los clientes
    \item Determinar la posición de los competidores en la mente de los clientes e intermediarios
    \item Revisar las barreras que impiden tener la oportunidad de participar en el mercado
    \item Considerar las amenazas actuales y potenciales así como las oportunidades, inluendo los
    avances tencológicos, los cambios arancelarios, de los capitales, la mano de obra, la energía
    y los demás costos de insumos. Las exigencias actuales de anticontaminación y otros factores
    que podrían afectar la competitividad.
    \item Investigar a los posibles competidores del área manufacturera a la cual pretendemos
    ingresar y ganar un segmento del mercado
    \item Conocer las necesidades establecidas por las futuras empresas competidoras
    \item Recopilar, organizar y estudiar la información
    \item Establecer las estrategias indicadas de acuerdo al estudio de la información, y
posteriormente observar el comportamiento de nuestra empresa
\end{itemize}
\section{Fortalezas, oportunidades, debilidades y amenazas}
El análisis FODA es una herramienta estratégica para conocer la situación de una empresa,
identifica las amenazas y oportuidades que surgen del ambiente; y las fortalezas y debilidades
internas de la organización.
\section{Conceptos básicos asociados}%, definición, características
Para los fines generales de la ciencia Econmica, puede concebirse un mercado como un lugar
fisico donde se compra y se vende, donde en un momento dado solo hay un precio para un producto
dado. Este concepto puede incluir mercados como los de cualquier gran ciudad, y también las
grandes bolsas de articulos y valores. El mercado ha sido definido como el área en el cual
convergen las fuerzas de la oferta y la demanda.
\section*{Importanica}
Al efectuar un estudio de mercado este nos permite determinar ciertos valores a cerca del producto
propuesto.
\begin{itemize}
    \item Determinar las características del bien a producir
    \item Determinar las características y delimitar el área del mercado potencial
    \item Determinar el uso del bien a producir
    \item Determinar los subproductos si los hubiera
\end{itemize}
\subsection{Tipos de mercado}
\begin{itemize}
    \item Competencia pura
    \item Monopolio puro
    \item Oligopolio
    \item La competencia monopolística
\end{itemize}