\chapter{Proyectos}
\pagenumbering{arabic}
\setcounter{page}{1}
\section{Definición de proyectos}
Al hablar de proyectos se tiene en mente algo estructurado técnicamente. Se considera que la
preparación y evaluación de proyectos busca recopilar, crear y analizar de una forma sistemática
un conjunto de antecedentes económicos que nos permitan juzgar cualitativa y cuantitativamente
las ventajas y desventajas de asignar recursos a una determinada iniciativa. Se estima que
un proyecto es la indagación de una solución inteligente al planteamiento de un problema
tendiente a resolver entre otras, una necesidad humana
\subsection{Tipos de proyectos}
Se puede afirmar que los proyectos industriales son de:
\begin{itemize}
    \item Implantación
    \item Introducción de una nueva unidad de producción
    \item Extensión
    \item Expansión de instalaciones productivas
    \item Renovación
    \item Por obsolescencia de instalaciones y equipos manteniendo la misma capacidad
    \item Reubicación
    \item Por modificación de los precios de los componentes
\end{itemize}
Se considera qie los proyectos privados surgen como respuesta a dos tipos de estimulos:
\begin{itemize}
    \item Se tiene presencia de un mercado amplio y en crecimiento
    \item Por la presencia de estímulos financieros, fiscales y cambiarios establecidos por
las autoridades gubernamentales. Esto en beneficio de ciertas áreas preferenciales de inversión,
ya sea en términos sectoriales y regionales
\end{itemize}
\subsection{Origen de proyectos}
Un proyecto de inversión puede surgir a causa de las siguientes circunstancias:
\begin{itemize}
    \item Por la realización de estudios sectoriales
    \item De un programa global de desarrollo
    \item Sustitución de importaciones
    \item Crecimiento de la demanda interna
    \item Demanda insatisfecha
\end{itemize}